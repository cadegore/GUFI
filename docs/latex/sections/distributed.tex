% This file is part of GUFI, which is part of MarFS, which is released
% under the BSD license.
%
%
% Copyright (c) 2017, Los Alamos National Security (LANS), LLC
% All rights reserved.
%
% Redistribution and use in source and binary forms, with or without modification,
% are permitted provided that the following conditions are met:
%
% 1. Redistributions of source code must retain the above copyright notice, this
% list of conditions and the following disclaimer.
%
% 2. Redistributions in binary form must reproduce the above copyright notice,
% this list of conditions and the following disclaimer in the documentation and/or
% other materials provided with the distribution.
%
% 3. Neither the name of the copyright holder nor the names of its contributors
% may be used to endorse or promote products derived from this software without
% specific prior written permission.
%
% THIS SOFTWARE IS PROVIDED BY THE COPYRIGHT HOLDERS AND CONTRIBUTORS "AS IS" AND
% ANY EXPRESS OR IMPLIED WARRANTIES, INCLUDING, BUT NOT LIMITED TO, THE IMPLIED
% WARRANTIES OF MERCHANTABILITY AND FITNESS FOR A PARTICULAR PURPOSE ARE DISCLAIMED.
% IN NO EVENT SHALL THE COPYRIGHT HOLDER OR CONTRIBUTORS BE LIABLE FOR ANY DIRECT,
% INDIRECT, INCIDENTAL, SPECIAL, EXEMPLARY, OR CONSEQUENTIAL DAMAGES (INCLUDING,
% BUT NOT LIMITED TO, PROCUREMENT OF SUBSTITUTE GOODS OR SERVICES; LOSS OF USE,
% DATA, OR PROFITS; OR BUSINESS INTERRUPTION) HOWEVER CAUSED AND ON ANY THEORY OF
% LIABILITY, WHETHER IN CONTRACT, STRICT LIABILITY, OR TORT (INCLUDING NEGLIGENCE
% OR OTHERWISE) ARISING IN ANY WAY OUT OF THE USE OF THIS SOFTWARE, EVEN IF
% ADVISED OF THE POSSIBILITY OF SUCH DAMAGE.
%
%
% From Los Alamos National Security, LLC:
% LA-CC-15-039
%
% Copyright (c) 2017, Los Alamos National Security, LLC All rights reserved.
% Copyright 2017. Los Alamos National Security, LLC. This software was produced
% under U.S. Government contract DE-AC52-06NA25396 for Los Alamos National
% Laboratory (LANL), which is operated by Los Alamos National Security, LLC for
% the U.S. Department of Energy. The U.S. Government has rights to use,
% reproduce, and distribute this software.  NEITHER THE GOVERNMENT NOR LOS
% ALAMOS NATIONAL SECURITY, LLC MAKES ANY WARRANTY, EXPRESS OR IMPLIED, OR
% ASSUMES ANY LIABILITY FOR THE USE OF THIS SOFTWARE.  If software is
% modified to produce derivative works, such modified software should be
% clearly marked, so as not to confuse it with the version available from
% LANL.
%
% THIS SOFTWARE IS PROVIDED BY LOS ALAMOS NATIONAL SECURITY, LLC AND CONTRIBUTORS
% "AS IS" AND ANY EXPRESS OR IMPLIED WARRANTIES, INCLUDING, BUT NOT LIMITED TO,
% THE IMPLIED WARRANTIES OF MERCHANTABILITY AND FITNESS FOR A PARTICULAR PURPOSE
% ARE DISCLAIMED. IN NO EVENT SHALL LOS ALAMOS NATIONAL SECURITY, LLC OR
% CONTRIBUTORS BE LIABLE FOR ANY DIRECT, INDIRECT, INCIDENTAL, SPECIAL,
% EXEMPLARY, OR CONSEQUENTIAL DAMAGES (INCLUDING, BUT NOT LIMITED TO, PROCUREMENT
% OF SUBSTITUTE GOODS OR SERVICES; LOSS OF USE, DATA, OR PROFITS; OR BUSINESS
% INTERRUPTION) HOWEVER CAUSED AND ON ANY THEORY OF LIABILITY, WHETHER IN
% CONTRACT, STRICT LIABILITY, OR TORT (INCLUDING NEGLIGENCE OR OTHERWISE) ARISING
% IN ANY WAY OUT OF THE USE OF THIS SOFTWARE, EVEN IF ADVISED OF THE POSSIBILITY
% OF SUCH DAMAGE.



\subsection{Distributed Processing}
\label{sec:distributed}

Filesystem trees are easily composable by simply placing multiple
(sub)trees under the same parent directory. This concept can be
inverted to allow for many GUFI operations to be distributed across
multiple nodes. This capability has become required due to resource
utilization issues when running GUFI at scale.

\texttt{gufi\_distributed.py} has been created to provide a framework
for distributing GUFI operations across multiple threads. How this
framework works at a high level is as follows: view a filesystem tree
as approximately pyramidal. The root is a single point at the very
top, and subdirectories fan out the further down one is. Slice the
pyramid horizontally at a fixed level. Each directory at and below the
slice level is a work item for one of the worker nodes. The top of the
pyramid, root to level - 1, will be processed by another worker. This
framework makes distributing GUFI operations into a balancing act of
distributing approximately the same number of directories across all
worker nodes to maximize parallelism.

More formally:

\begin{enumerate}
\item Choose a level in the tree to split up across nodes.
\item Write the directory paths (without the starting path) at the
  selected level to a file. The generic POSIX command would be:
  \texttt{find <path> -mindepth <level> -maxdepth <level> -type d
    -printf "\%P\textbackslash n"}. If a specific filesystem has a
  faster tool for finding paths, that tool should be used instead.
\item Let \texttt{W > 1} be the number of worker nodes used to process
  the tree. \texttt{S~=~W~-~1} nodes will be used to process the
  subtrees. 1 node will process the top.
\item Distribute the directories from the previous step to \texttt{S}
  separate files. Each file will be the work a single node processes.
\item On the \texttt{S} subtree worker nodes, run the executable with
  \texttt{-y level} and \texttt{-D <filename>}. All other flags and
  arguments used should remain the same.
\item On the final worker node, run the executable with \texttt{-z
  (level - 1)} to process the directories above the ones that were
  split across nodes.
\end{enumerate}

\texttt{gufi\_distributed.py} allows for work to be distributed via
\texttt{ssh} and \texttt{sbatch}.

\subsubsection{Distributed Operations}
The following operations have scripts that distribute processing:

\begin{itemize}
\item \gufidirindex
\item \gufidirtrace
\item \gufitreesummaryall
\item \gufirollup
\item \gufiunrollup
\item \gufiquery
\end{itemize}

These scripts are found in \texttt{scripts/distributed} in the source
and build directories, and are installed as
\texttt{<operation>\_distributed}.

\paragraph{Notes}
\begin{itemize}
\item The target path of an indexing operation does not need to be a
  shared location, as long as the root path on each worker node is the
  same. This works because trees are composable, so there is no
  difference between a tree with subdirectories \texttt{dir1} and
  \texttt{dir2}, and two separate trees with one of \texttt{dir1} or
  \texttt{dir2}. The only practical difference when querying would be
  more warnings due to missing \texttt{db.db} files in the parents of
  the starting points.
\item Querying has an additional steps of requiring results to be
  written to per-worker-node file(s) and at the end, combining the
  results and processing them one last time. This is currently an
  action the user must do themselves.
\end{itemize}
